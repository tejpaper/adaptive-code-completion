\chapter{Code Completion}\label{chap:code-completion}

The upcoming chapter provides a comprehensive examination of code completion. It begins with a definition of the task, then delves into the motivations for advancing code completion techniques, highlighting their impact on productivity and learning. Finally, the categorization of code completion based on various criteria is presented.

\section{Task Definition}

\textit{Code completion}, also called \textit{code suggestion} or \textit{autocompletion}, is a functionality within integrated development environments that enables developers to utilize an automated continuation of the code they are writing. Typically, it is invoked by a trigger model, which aims to anticipate the user's requirement for this functionality. The point of this invocation, whether it is referenced as temporal or spatial, is called a \textit{trigger point}.

In this work, code completion is regarded as a task rather than a feature, as this perspective highlights the task-methodology relationship and emphasizes the technical aspects over marketing considerations. In addition, the term \textit{completion} is used as a shorthand throughout the text.

\section{Motivation}

The task of code completion is a pivotal component of contemporary software development, offering substantial benefits that enhance both the productivity and efficiency of developers. This section elucidates the motivations for investigating this task and advancing existing solutions.

By reducing the amount of typing required, code completion allows developers to focus more on the logic and structure of their code rather than the syntax. Studies have shown that AI-powered code completion tools can significantly reduce task execution times, with some reporting up to a 55.8\% reduction in controlled experiments \parencite{peng2023}. These tools also significantly impact developer productivity by reducing the cognitive load associated with coding tasks, achieved through contextually relevant suggestions that streamline the coding process \parencite{weber2024}. Furthermore, the integration of AI in code completion tools has been linked to increased developer satisfaction and efficiency, as it allows developers to complete tasks more swiftly and with greater accuracy \parencite{bakal2025}.

For students and novice programmers, code completion tools serve as a valuable learning aid. They provide real-time feedback and suggestions, helping learners understand programming concepts and syntax more effectively. This educational aspect is highlighted in studies where students reported enhanced learning experiences and increased confidence in their coding abilities \parencite{takerngsaksiri2023}. Moreover, code completion tools provide accurate suggestions that help reduce typo errors and other common mistakes. This is particularly beneficial for new developers or those working with unfamiliar codebases, as it helps them adhere to coding standards and best practices.

Recent advancements in deep learning have enabled the personalization of code completion tools, allowing them to adapt to specific organizational or individual coding styles. This personalization not only improves the relevance of suggestions but also enhances the overall user experience, making these tools more effective and user-friendly \parencite{giagnorio2025}.

Due to the simple and atomic formulation of code completion, it possesses the important property of being fundamental to other AI-powered code assistance features. For instance, code generation can be viewed as a specific extension of code completion, and code editing is a sequence of code generations. This implies that most enhancements achieved through code completion research propagate further, adding new layers of improvements in the field.

In summary, code completion stands as a foundational and multifaceted tool within modern software development. Its benefits extend beyond mere convenience, offering measurable improvements in efficiency, learning, and code quality. As advancements in AI continue to refine its capabilities, code completion not only streamlines day-to-day programming tasks but also acts as a cornerstone for broader innovations in AI-assisted software engineering.

\section{Taxonomy of Code Completion}

The concept of code completion lacks a strict definition within the field, as its interpretation has evolved with the various approaches employed to address it over time (see the Related Work section of the~\citet{ciniselli2021} for more details). This section provides a comprehensive overview of the diverse characteristics of code completion and highlights the specific focus of this thesis.

To grasp the meaning of the subsequent text, it is crucial to understand the term \textit{token}, which refers to any subsequence of characters with a finite length. The utility of this term is further justified in \sectionref{sec:tokens}.

\subsection{Granularity}

Completion can be applied under the different degrees of granularity, which refers to the scope and detail of the produced code. At the most basic level, \textit{next-token completion} involves predicting the subsequent token, such as a keyword, operator, or identifier. This level of granularity is useful for fine-grained suggestions that assist developers in writing code efficiently. Moving up in complexity, \textit{single line completion} involves predicting the continuation of the given line of code based on the current context, which requires a broader understanding of the code's structure and logic. At the highest level, \textit{code block completion} involves generating entire blocks of code, such as functions or classes, which necessitates a comprehensive understanding of the codebase and its architecture. This level of granularity is akin to code generation, where the model not only completes existing code but also creates new, coherent code structures. Each level of granularity serves different purposes and can be leveraged depending on the specific needs of the development task at hand.

For this work, single-line completion is selected because it presents a greater challenge for modern models compared to next-token prediction and serves as a fundamental baseline task for evaluating the performance of the proposed methods.

\subsection{Context}

In the realm of code completion, a distinction is drawn between file-level and repository-level approaches, each characterized by its scope and contextual depth. \textit{File-level code completion} operates within the confines of a single file, leveraging the immediate context such as local variable declarations, function definitions, and imports. This approach is effective for small, isolated scripts but is limited in its ability to capture the broader interactions present in larger projects. Conversely, \textit{repository-level code completion} extends its reach to encompass the entire codebase, integrating information from multiple files and modules. This broader context is essential for understanding complex dependencies and interactions that span across the repository, enabling more accurate and contextually relevant code suggestions. The need for greater context in repository-level completion is driven by the intricate and interconnected nature of modern software systems, where a comprehensive understanding of the entire codebase is crucial for effective code completion.

Repository-level completion has been selected as the focus of this thesis because it offers a broader research landscape and addresses a significant need within the field.

\subsection{Suffix Awareness}

The completion scenario can be categorized based on the directionality of the prediction relative to the trigger point. The \textit{left-to-right} scenario involves generating code predictions using only the context preceding the trigger point, which is typical in traditional code completion systems. This approach is effective for straightforward code continuations but may lack the ability to consider the broader context. Conversely, the \textit{left-and-right} scenario, also known as \textit{bidirectional} completion, leverages both the preceding and succeeding context around the trigger point. This method allows for more informed predictions by considering the entire file, thus enhancing the accuracy and relevance of the suggestions.

Modern models for code completion employ the Fill-in-the-Middle (FIM) approach to gain suffix awareness capability. Although this thesis describes this method in \sectionref{sec:fill-in-the-middle}, it does not expand on it and instead focuses on the more fundamental left-to-right scenario. This choice is driven by the practical part of this work and its design constraints, which are motivated in \sectionref{sec:training}.

\subsection{Line Start Availability}

Another criterion for categorizing completion systems is whether the initial portion of the target line has already been completed by the user. In practical applications, this scenario is quite common. However, for the purpose of model evaluation, it is more straightforward to assess the \textit{full line completion}, where the initial part of the target line is not provided. This approach serves as a lower bound for assessing completion quality, as the ability to predict the entire line demonstrates a stronger capability.

Due to the aforementioned reasoning, the full line code completion is further considered throughout this thesis.

\subsection{Programming Language Support}

The code completion systems can support either a single programming language or multiple languages simultaneously.

In the practical part of this thesis, multilingual models are employed with a focus on a single programming language, specifically Python\footnote{\url{https://www.python.org/}}. This design choice is motivated by resource constraints.
\medskip

To finalize the terminology used in subsequent parts of this thesis regarding this task, the \textit{completion} refers to a full, single-line, repository-level code completion without suffix awareness.
